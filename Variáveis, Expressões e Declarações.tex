%%%%%%%%%%%%%%%%%%%%%%%%%%%%%%%%%%%%%%%%%
% Short Sectioned Assignment
% LaTeX Template
% Version 1.0 (5/5/12)
%
% This template has been downloaded from:
% http://www.LaTeXTemplates.com
%
% Original author:
% Frits Wenneker (http://www.howtotex.com)
%
% License:
% CC BY-NC-SA 3.0 (http://creativecommons.org/licenses/by-nc-sa/3.0/)
%
%%%%%%%%%%%%%%%%%%%%%%%%%%%%%%%%%%%%%%%%%

%----------------------------------------------------------------------------------------
%	PACKAGES AND OTHER DOCUMENT CONFIGURATIONS
%----------------------------------------------------------------------------------------

\documentclass[paper=a4, fontsize=11pt]{scrartcl} % A4 paper and 11pt font size

\usepackage[T1]{fontenc} % Use 8-bit encoding that has 256 glyphs
\usepackage[brazil] {babel}
\usepackage{fourier} % Use the Adobe Utopia font for the document - comment this line to return to the LaTeX default
%\usepackage[english]{babel} % English language/hyphenation
\usepackage{amsmath,amsfonts,amsthm} % Math packages
\usepackage{graphicx}

\usepackage{lipsum} % Used for inserting dummy 'Lorem ipsum' text into the template

\usepackage{sectsty} % Allows customizing section commands
%\allsectionsfont{\centering \normalfont\scshape} % Make all sections centered, the default font and small caps
%\allsectionsfont{\normalfont\scshape} % Make all sections centered, the default font and small caps

\usepackage{fancyhdr} % Custom headers and footers
\pagestyle{fancyplain} % Makes all pages in the document conform to the custom headers and footers
\fancyhead{} % No page header - if you want one, create it in the same way as the footers below
\fancyfoot[L]{} % Empty left footer
\fancyfoot[C]{} % Empty center footer
\fancyfoot[R]{\thepage} % Page numbering for right footer
\renewcommand{\headrulewidth}{0pt} % Remove header underlines
\renewcommand{\footrulewidth}{0pt} % Remove footer underlines
\setlength{\headheight}{13.6pt} % Customize the height of the header

\numberwithin{equation}{section} % Number equations within sections (i.e. 1.1, 1.2, 2.1, 2.2 instead of 1, 2, 3, 4)
\numberwithin{figure}{section} % Number figures within sections (i.e. 1.1, 1.2, 2.1, 2.2 instead of 1, 2, 3, 4)
\numberwithin{table}{section} % Number tables within sections (i.e. 1.1, 1.2, 2.1, 2.2 instead of 1, 2, 3, 4)

\setlength\parindent{0pt} % Removes all indentation from paragraphs - comment this line for an assignment with lots of text

%----------------------------------------------------------------------------------------
%	TITLE SECTION
%----------------------------------------------------------------------------------------

\newcommand{\horrule}[1]{\rule{\linewidth}{#1}} % Create horizontal rule command with 1 argument of height

\title{	
\normalfont \normalsize 
\textsc{Introdução à Programação de Computadores -- DCC/UFMG} \\ [25pt] % Your university, school and/or department name(s)
\horrule{2pt} \\[0.4cm] % Thin top horizontal rule
\huge Aula Prática 1 \\ % The assignment title
\vspace{0.25cm}
%\large Data de entrega: \textbf{até às 23:59 de 11/01/2022} \\ % The assignment title
\horrule{2pt} \\[0.5cm] % Thick bottom horizontal rule
}

%\author{Adriano César Machado Pereira\\João Guilherme Maia de Menezes} % Your name

%\date{\normalsize\today} % Today's date or a custom date
\date{}

\begin{document}

\maketitle % Print the title

%----------------------------------------------------------------------------------------
%	PROBLEM
%----------------------------------------------------------------------------------------


\small
\section*{Instruções para Submissão}

Na aula prática de hoje, você terá que elaborar programas para resolver problemas diversos, conforme descrito abaixo. Cada uma das soluções deverá ser implementada em seu próprio arquivo com extensão \texttt{.py}. Por exemplo, a solução para o problema 1 deverá ser implementada em um arquivo chamado \texttt{problema1.py}, a solução para o problema 2 deverá ser implementada no arquivo \texttt{problema2.py} e assim por diante. Finalmente, submeta cada um dos arquivos pelo Moodle.\\

\textbf{Dica:} se você tiver problemas com caracteres especiais (caracteres com acentos, por exemplo), adicione a linha abaixo na primeira linha de todos os arquivos \texttt{*.py}\\

\texttt{\# -*- coding: utf-8 -*-}




\section*{Problema 1}

Faça um programa que leia dois catetps de um triângulo retângulo e imprima na tela o valor da hipotenusa desse triângulo.\\

\textbf{Observação:} as mensagens exibidas para o usuário deverão ser exatamente como apresentado abaixo (mensagens exibidas com os comandos \texttt{input()} e \texttt{print()}).\\


\textbf{Exemplo de execução do programa:}\\

Digite o valor do cateto a: \textbf{20.15} \\
Digite o valor do cateto b: \textbf{12.11} \\

Hipotenusa = \textbf{23.51} \\




\section*{Problema 2}

Escreva um programa que leia a altura de uma pirâmide de base quadrada e a medida dos lados do polígono da base e calcule o volume dessa pirâmide.\\

\textbf{Observação:} as mensagens exibidas para o usuário deverão ser exatamente como apresentado abaixo (mensagens exibidas com os comandos \texttt{input()} e \texttt{print()}).\\

\textbf{Exemplo de execução do programa:}\\

Digite a altura da pirâmide: \textbf{10} \\
Digite o valor do lado do quadrado na base da pirâmide: \textbf{23} \\

Volume da pirâmide: \textbf{1763.33} \\


\section*{Problema 3}

Um professor de mergulho profissional deseja tornar mais prático para os alunos entenderem a relação entre a profundidade mergulhada e a pessão exercida pela água. Escreva um programa que leia a profundidade que o aluno deseja mergulhar e imprima na tela a pressão exercida sobre ele em termos da pressão atmosférica. Considere a aceleração da gravidade como 10 m/s², a densidade da água como 1000 kg/m³ e a pressão atmosférica como 101000 Pa.\\

\begin{equation*}
    P = \rho g h + P_{atm}
\end{equation*}

\textbf{Observação:} as mensagens exibidas para o usuário deverão ser exatamente como apresentado abaixo (mensagens exibidas com os comandos \texttt{input()} e \texttt{print()}).\\

\textbf{Exemplo de execução do programa:}\\

Digite a altura (em metros) que deseja mergulhar: \textbf{50} \\

Pressão exercida pela água: \textbf{5.95 $P_{atm}$} \\



\section*{Problema 4}

A temperatura média de uma região durante a primavera é 25 ºC. Faça um programa que receba a temperatura média de cada mês dessa estação e calcule a média entre essas temperaturas. Após isto, imprima na tela qual foi a variação de temperatura naquele ano com relação à temperatura média natural da região nessa estação.\\

\textbf{Observação:} as mensagens exibidas para o usuário deverão ser exatamente como apresentado abaixo (mensagens exibidas com os comandos \texttt{input()} e \texttt{print()}).\\

\textbf{Exemplo de execução do programa:}

Digite a temperatura média em setembro: \textbf{35} \\
Digite a temperatura média em outubro: \textbf{31} \\
Digite a temperatura média em novembro: \textbf{23} \\

Variação de Temperatura: \textbf{4.67 ºC}


\section*{Problema 5}

Uma empresa deseja calcular sua média salarial. Nessa empresa, existem quatro cargos: Auxiliar administrativo, Vendedores, Supervisores e Gerente de vendas. Escreva um programa leia os salários e o número de funcionários correspondentes a cada cargo e imprima na tela a média salarial dessa empresa.\\

\textbf{DICA:} Utilize média ponderada para resolver o problema. \\

\textbf{Observação:} as mensagens exibidas para o usuário deverão ser exatamente como apresentado abaixo (mensagens exibidas com os comandos \texttt{input()} e \texttt{print()}).\\

\textbf{Exemplo de execução do programa:}\\

Digite o valor do salário para Auxiliar administrativo: \textbf{1800.00} \\
Digite o número de Auxiliares administrativos: \textbf{2} \\
Digite o valor do salário para Vendedor: \textbf{2500.00} \\
Digite o número de Vendedores: \textbf{26} \\
Digite o valor do salário de Supervisor: \textbf{5000.00} \\
Digite o número de Supervisores: \textbf{8} \\
Digite o valor do salário para Gerente de vendas: \textbf{8000.00} \\
Digite o número de Gerentes de vendas: \textbf{4} \\

Média salarial: \textbf{3515.00}
\end{document}
